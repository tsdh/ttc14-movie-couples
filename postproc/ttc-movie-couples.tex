\documentclass[submission]{eptcs}
\providecommand{\event}{TTC 2014}

\usepackage[T1]{fontenc}
\usepackage{url}
\usepackage{hyperref}
\usepackage{geometry}
\usepackage{paralist}
\usepackage[cache]{minted}
\newminted{clojure}{fontsize=\fontsize{8}{8},linenos,numbersep=3pt}
\newmintinline{clojure}{fontsize=\fontsize{8}{8}}
\newcommand{\code}{\clojureinline}

\title{Solving the TTC Movie Database Case with FunnyQT}
\author{Tassilo Horn
  \institute{Institute for Software Technology, University Koblenz-Landau, Germany}
  \email{horn@uni-koblenz.de}}

\def\titlerunning{Solving the TTC Movie Database Case with FunnyQT}
\def\authorrunning{T. Horn}

\begin{document}
\maketitle

\begin{abstract}
  FunnyQT is a model querying and model transformation library for the
  functional Lisp-dialect Clojure providing a rich and efficient querying and
  transformation API.  This paper describes the FunnyQT solution to the TTC
  2014 Movie Database transformation case.  All core tasks and all extension
  tasks have been solved.
\end{abstract}


\section{Introduction}
\label{sec:introduction}

This paper describes a solution of the TTC 2014 Movie Database
Case~\cite{movies-case-desc}.  All core and extension tasks have been solved,
and the solution also scales well with large models.  The solution project is
available on
Github\footnote{\url{https://github.com/tsdh/ttc14-movie-couples}}, and it is
set up for easy reproduction on the
SHARE\footnote{\url{http://is.ieis.tue.nl/staff/pvgorp/share/}} image
\texttt{Ubuntu12LTS\_TTC14\_64bit\_FunnyQT4.vdi}.

The solution is implemented using FunnyQT~\cite{Horn2013MQWFQ} which is a model
querying and transformation library for the functional Lisp dialect
Clojure\footnote{\url{http://clojure.org}}.  Queries and transformations are
plain Clojure programs using the features provided by the FunnyQT API.  This
API is structured into several task-specific sub-APIs/namespaces, e.g., there
is a namespace \emph{funnyqt.in-place} containing constructs for writing
in-place transformations, a namespace \emph{funnyqt.model2model} containing
constructs for model-to-model transformations, a namespace \emph{funnyqt.bidi}
containing constructs for bidirectional transformations, and so forth.

As a Lisp dialect, Clojure provides strong metaprogramming capabilities that
are exploited by FunnyQT in order to define several \emph{embedded
  domain-specific languages} (DSL, \cite{book:Fowler2010DSL}) for different
tasks.  For example, the pattern matching, in-place transformation,
model-to-model transformation, and bidirectional model transformation
constructs are provided in terms of a small, task-oriented DSL each.

FunnyQT currently supports querying and transforming
EMF~\cite{Steinberg2008EEM} and
JGraLab\footnote{\url{http://jgralab.uni-koblenz.de}} TGraph models, and
support for other modeling frameworks can be added without touching FunnyQT's
internals.  For both EMF and JGraLab, there is one FunnyQT core namespace
(i.e., \emph{funnyqt.emf} and \emph{funnyqt.tg}) providing functions for
accessing and manipulation models of that kind using the framework's
terminology and giving access to any feature provided by that framework.  These
core namespaces are complemented by a namespace \emph{funnyqt.generic} which
provides functions for functionality available on all frameworks in a generic,
framework-agnostic manner.


\section{Solution Description}
\label{sec:solution-description}

In this section, the complete transformation and query specifications for all
core and extension tasks are going to be explained.  In the listings given in
the following, all function calls are shown in a namespace-qualified form to
make it explicit in which Clojure or FunnyQT namespace those functions are
defined.  Clojure allows to define short aliases for used namespaces in order
to allow qualification while still being concise.  Table~\ref{tab:namespaces}
gives an overview of all namespaces used by the solution, and the aliases used
for accessing them.

\begin{table}[h!t]
  \centering
  \begin{tabular}{| l | l | l |}
    \hline
    \textbf{Alias} & \textbf{Namespace} & \textbf{Description}\\
    \hline
    \textsf{emf}  & \textsf{funnyqt.emf}      & Core EMF API\\
    \textsf{gen}  & \textsf{funnyqt.generic}  & Generic model access functions\\
    \textsf{io}   & \textsf{clojure.java.io}  & File IO functions\\
    \textsf{ip}   & \textsf{funnyqt.in-place} & In-place transformation API\\
    \textsf{poly} & \textsf{funnyqt.polyfns}  & Polymorphic function API\\
    \textsf{set}  & \textsf{clojure.set}      & Set functions (e.g., union, intersection,...)\\
    \textsf{str}  & \textsf{clojure.string}   & String utility functions\\
    \textsf{u}    & \textsf{funnyqt.utils}    & Utility functions (e.g., error handling)\\
    \hline
  \end{tabular}
  \caption{Used Clojure and FunnyQT namespaces with their aliases}
  \label{tab:namespaces}
\end{table}

All function calls that are not qualified with an alias are calls to functions
in the \emph{clojure.core} namespace which is available in any other namespace
by default.

\subsection{Task 1: Generating Test Data}
\label{sec:task-1:generating-test-data}

The first task is generating test data.  The case
description~\cite{movies-case-desc} illustrates the task with Henshin rules.
Since the rules actually don't match anything but simply create new elements in
the model, we have implemented them as plain functions receiving the model and
an integer parameter \code|i| from which the movie ratings and actor names are
derived.

The function \code|create-positive!|\footnote{The \code|^:private| is a
  metadata annotation declaring that this function is private, i.e., it is not
  accessible from another namespace.} creates five movies, three actors, and
two actresses.  The persons' \textsf{movies} references are set as requested by
the case description, i.e., every persons acts in the second, third, and fourth
movie, and the first two persons additionally act in the first movie while the
last two persons additionally act in the fifth movie.

Forward-looking to task 2, every invocation of the \code|create-positive!|
function creates \({5 \choose 2} = 10\) couples with three common movies.

\begin{clojurecode}
(defn ^:private create-positive! [model i]
  (let [m1 (emf/ecreate! model 'Movie :rating (+ 0.0 (* 10 i)))
        m2 (emf/ecreate! model 'Movie :rating (+ 1.0 (* 10 i)))
        m3 (emf/ecreate! model 'Movie :rating (+ 2.0 (* 10 i)))
        m4 (emf/ecreate! model 'Movie :rating (+ 3.0 (* 10 i)))
        m5 (emf/ecreate! model 'Movie :rating (+ 4.0 (* 10 i)))]
    (emf/ecreate! model 'Actor   :name (str "a" (* 10 i))       :movies [m1 m2 m3 m4])
    (emf/ecreate! model 'Actor   :name (str "a" (+ 1 (* 10 i))) :movies [m1 m2 m3 m4])
    (emf/ecreate! model 'Actor   :name (str "a" (+ 2 (* 10 i))) :movies [m2 m3 m4])
    (emf/ecreate! model 'Actress :name (str "a" (+ 3 (* 10 i))) :movies [m2 m3 m4 m5])
    (emf/ecreate! model 'Actress :name (str "a" (+ 4 (* 10 i))) :movies [m2 m3 m4 m5])))
\end{clojurecode}

The \code|create-negative!| function is defined similarly.  It also creates
five movies and five persons assigning ratings for the former and names for the
latter.  However, every pair of two of those persons have acted together in at
most two movies, so there is no couple to be found in task 2 here.

\begin{clojurecode*}{firstnumber=12}
(defn ^:private create-negative! [model i]
  (let [m1 (emf/ecreate! model 'Movie :rating (+ 5.0 (* 10 i)))
        m2 (emf/ecreate! model 'Movie :rating (+ 6.0 (* 10 i)))
        m3 (emf/ecreate! model 'Movie :rating (+ 7.0 (* 10 i)))
        m4 (emf/ecreate! model 'Movie :rating (+ 8.0 (* 10 i)))
        m5 (emf/ecreate! model 'Movie :rating (+ 9.0 (* 10 i)))]
    (emf/ecreate! model 'Actor   :name (str "a" (+ 5 (* 10 i))) :movies [m1 m2])
    (emf/ecreate! model 'Actor   :name (str "a" (+ 6 (* 10 i))) :movies [m1 m2 m3])
    (emf/ecreate! model 'Actress :name (str "a" (+ 7 (* 10 i))) :movies [m2 m3 m4])
    (emf/ecreate! model 'Actress :name (str "a" (+ 8 (* 10 i))) :movies [m3 m4 m5])
    (emf/ecreate! model 'Actress :name (str "a" (+ 9 (* 10 i))) :movies [m4 m5])))
\end{clojurecode*}

The last function \code|create-example| is the entry point to the generation of
synthetic test models.  It receives an integer parameter \code|n|, creates a
new model (a resource in EMF parlance), and then invokes the
\code|create-positive!| and \code|create-negative!| rules \code|n| times with
the new model and values for \code|i| from 0 to \code|n-1|.  Finally, the new
model populated with the movies and actors is returned.

\begin{clojurecode*}{firstnumber=23}
(defn create-example [n]
  (let [model (emf/new-resource)]
    (dotimes [i n]
      (create-positive! model i)
      (create-negative! model i))
    model))
\end{clojurecode*}

The \code|create-example| function is invoked by the benchmark runner with the
\code|n|-values 1000, 2000, 3000, 4000, 5000, 10000, 50000, 100000, and 200000.

\subsection{Task 2 \& 3 and Extension Task 2 \& 3: Finding Couples/Cliques \&
  Compute Average Rankings}
\label{sec:task-2:finding-groups}

As the case description already mentions, the \emph{Finding Cliques} task is a
generalization of the \emph{Finding Couples} task, so the FunnyQT solution
solves both one go.  For finding cliques of arbitrary sizes \(n \geq 3\),
a higher-order transformation should be defined that creates a transformation
rule for that \(n\).
The FunnyQT solution also allows for \(n = 2\)
and deals with the fact that in this case, \code|Couple| elements should be
created rather than \code|Clique| elements.

Also, the \emph{computation of the average rankings} of a couple's or clique's
common movies is done while creating the \code|Couple| or \code|Clique| element
instead of doing it separately in a further step.

The higher-order transformation generating a FunnyQT in-place transformation
rule for a given \(n \geq 2\)
is a Clojure \emph{macro}.  A macro is a function which is executed at
compile-time by the Clojure compiler.  It receives code passed to it as
arguments, processes it, and returns new code that takes the place of it's
call.  This new code is called the macro's \emph{expansion}.  Because like all
Lisps, Clojure is \emph{homoiconic}, i.e., Clojure code is represented using
Clojure datastructures\footnote{In other words, Lisp and Clojure do not
  separate concrete and abstract syntax.} (literals, symbols, lists, vectors),
a macro is essentially a transformation on the abstract syntax tree of the
Clojure code that's passed to the macro.

Before discussing the rule generation macro, a few helper functions are going
to be discussed.  Those will be used as constraints in the patterns of the
generated rules.

\begin{clojurecode}
(defn movie-count [p]
  (.size ^java.util.Collection (emf/eget-raw p :movies)))

(defn person-count [m]
  (.size ^java.util.Collection (emf/eget-raw m :persons)))

(defn movie-set [p]
  (into #{} (emf/eget-raw p :movies)))
\end{clojurecode}

The function \code|movie-count| gets some \code|Person| element \code|p| and
returns the number of movies that this person has acted in.
\code|emf/eget-raw| is a function that returns the value of some element's
property without any conversion\footnote{The usual FunnyQT EMF property getter
  \code|emf/eget| coerces Java collections to immutable, persistent Clojure
  collections.  Since we are only interested in the size, this coercion would
  be superfluous overhead.}, e.g., here an EMF \code|EList| object is returned
and its \code|size()| method (defined in the EMF Java API) is invoked.  The
syntax \code|^java.util.Collection| is a type hint which allows the Clojure
compiler to generate byte-code for a direct rather than a reflective Java
method call (\code|EList| implements the \code|Collection| interface which
declares the method \code|size()|).

Similarly, \code|person-count| returns the number of persons that acted in a
given movie \code|m|.

Lastly, \code|movie-set| gets a person element \code|p| and returns the movies
that person acted in as a set.  \code|#{}| is the empty Clojure set literal.

The \code|avg-rating| function shown in the next listing gets a collection of
movie elements and returns their average rating.  The higher-order function
\code|map| takes a function and a collection and applies the function to each
element of the collection returning the sequence of results, which is the
sequence of the given movies ratings here.  \code|reduce| then aggregates the
sequence using \code|+|, i.e., it computes the sum of all ratings.  Finally,
this sum is divided by the number of given movies.

\begin{clojurecode*}{firstnumber=9}
(defn avg-rating [movies]
  (/ (reduce + (map #(emf/eget % :rating) movies))
     (count movies)))
\end{clojurecode*}

The function \code|n-common-movies?| printed in the next listing gets an
integer \code|n|, a person element \code|p|, and additional person elements
\code|more|\footnote{The Clojure varargs syntax \code|& els| is similar to
  Java's \code|Type... els| syntax.}.  If all given persons act together in at
least \code|n| movies, the set of common movies is returned.  Otherwise,
\code|nil| is returned.  Since in Clojure the values \code|nil| and
\code|false| are falsy while every other value is truthy, this function can act
as a predicate and still return more information, i.e., the common movies, in
the positive case.

\begin{clojurecode*}{firstnumber=12}
(defn n-common-movies? [n p & more]
  (loop [common (movie-set p), more more]
    (when (>= (count common) n)
      (if (seq more)
        (recur (set/intersection common (movie-set (first more)))
               (rest more))
        common))))
\end{clojurecode*}

The next listings shows the \code|define-groups-rule| macro which is the
higher-order transformation solving the task.  It receives an parameter
\code|n| and, as its name suggests, expands into a rule for finding couples (if
\code|n| equals 2) or cliques of size \code|n|.

\begin{clojurecode*}{firstnumber=19}
(defmacro define-group-rule [n]
  (let [psyms (map #(symbol (str "p" %)) (range n))]
    `(ip/defrule ~(symbol (str "make-groups-of-" n "!"))
       {:forall true :no-result-vec true}
       [~'model ~'c]
       [~'m<Movie>
        :when (>= (person-count ~'m) ~n)
        ~@(mapcat (fn [i]
                    (let [ps (nth psyms i)]
                      `[~'m -<persons>-> ~ps
                        :when (>= (movie-count ~ps) ~'c)
                        ~@(when-not (zero? i)
                            `[:when (neg? (compare (emf/eget-raw ~(nth psyms (dec i)) :name)
                                                   (emf/eget-raw ~ps :name)))])
                        ~@(when-not (or (zero? i) (= i (dec n)))
                            `[:when (n-common-movies? ~'c ~@(take (inc i) psyms))])]))
                  (range n))
        :when-let [~'cms (n-common-movies? ~'c ~@psyms)]
        :as [~'cms ~@psyms]
        :distinct]
       (emf/ecreate! ~'model ~@(if (= n 2)
                                 `['Couple :p1 ~(first psyms) :p2 ~(second psyms)]
                                 `['Clique :persons [~@psyms]])
                     :commonMovies ~'cms :avgRating (avg-rating ~'cms)))))
\end{clojurecode*}

We're not going to discuss the macro in details, however the central idea of
the Clojure (or Lisp) macrosystem is that one defines the basic structure of
the macro's expansion using a \emph{quasi-quoted} (backticked) form as a kind
of template.  In this quasi-quoted form, values computed at compile-time can be
inserted using the \emph{unquote} (\code|~|) and \emph{unquote-splicing}
(\code|~@|) operators to fill the template's variable parts.  What the macro
above generates is a \code|(ip/defrule ...)| FunnyQT in-place transformation
rule definition where the rule's name is \code|make-groups-of-<n>!| (line 21),
where two special rule options are set (line 22), which has two parameters
\code|model| and \code|c| (line 23), a complex pattern involving constraints
using the \code|person-count|, \code|movie-count|, and \code|n-common-movies?|
functions defined above (lines 24-38), and which finally contains an action to
be applied to found matches that creates either a \code|Couple| or a
\code|Clique| and sets the common movies and average rating (lines 39-42).

The last part of the implementation of the tasks 2 and 3 and the extension
tasks 2 and 3 is to actually invoke the macro to create the transformation
rules for couples and cliques of 3, 4, and 5 persons.

\begin{clojurecode*}{firstnumber=43}
(define-group-rule 2) ;; make-groups-of-2!: The Couples rule
(define-group-rule 3) ;; make-groups-of-3!: The Cliques of Three rule
(define-group-rule 4) ;; make-groups-of-4!: The Cliques of Four rule
(define-group-rule 5) ;; make-groups-of-5!: The Cliques of Five rule
\end{clojurecode*}

Instead of discussing the rule generation macro in details, it makes more sense
to have an in-depth look at one of its expansion like the one for \code|n|
being 3 shown below.  A FunnyQT in-place transformation rule is defined whose
name is \code|make-groups-of-3!|, and it gets as arguments the \code|model| on
which it should be applied, and an integer \code|c| which determines how many
common movies a clique of three persons needs to have.  The case description
fixes \code|c| to 3, but with this parameter, we allow for a bit more
generality.

Lines 4 to 12 define the rule's pattern.  The structural part defines that it
matches a \code|Movie| element \code|m| which references three \code|Person|
elements \code|p0|, \code|p1|, and \code|p2| using its \code|persons|
reference.

Additionally, the pattern defines several constraints using the \code|:when|
keyword.  The movie \code|m| needs to have at least three acting persons (line
4), and all persons need to act in at least \code|c| movies (lines 5, 6, and
9).  To avoid duplicate matches where only the order of the three person
elements differs, the constraints in line 7 and 10 enforce a lexicographically
ascending order of the names of the persons \code|p0|, \code|p1|, and
\code|p2|.

{\renewcommand{\theFancyVerbLine}{\textcolor{blue}{\tiny\arabic{FancyVerbLine}}}
\begin{clojurecode}
(ip/defrule make-groups-of-3!
  {:forall true, :no-result-vec true}
  [model c]
  [m<Movie>          :when (>= (person-count m) 3)
   m -<persons>-> p0 :when (>= (movie-count p0) c)
   m -<persons>-> p1 :when (>= (movie-count p1) c)
                     :when (neg? (compare (emf/eget-raw p0 :name) (emf/eget-raw p1 :name)))
                     :when (n-common-movies? c p0 p1)
   m -<persons>-> p2 :when (>= (movie-count p2) c)
                     :when (neg? (compare (emf/eget-raw p1 :name) (emf/eget-raw p2 :name)))
   :when-let [cms (n-common-movies? c p0 p1 p2)]
   :as [cms p0 p1 p2] :distinct]
  (emf/ecreate! model 'Clique :persons [p0 p1 p2] :commonMovies cms
               :avgRating (avg-rating cms)))
\end{clojurecode}
}

Furthermore, line 8 ensures that \code|p0| and \code|p1| have at least \code|c|
common movies.  The same for the complete clique of three persons is also
asserted in line 11, where the common movies are also bound to the variable
\code|cms|.

The constraints in lines 4, 5, 6, 8, and 9 are not really needed.  Omitting
them would result is the very same set of matches.  However, such a FunnyQT
pattern is syntactic sugar for a search starting at \code|Movie| elements
\code|m| and iterating all combinations of the elements targeted by their
\code|persons| references, so it makes sense to add constraints as early as
possible in order to cut the search space.  Clearly, if a movie has less than 3
actors, it can't be part of a clique of three.  Likewise, persons that acted in
less than \code|c| movies can't be part of a clique that requires \code|c|
common movies.  And similarly, if \code|p0| and \code|p1| do not have the
required number of \code|c| common movies, then \code|p0|, \code|p1|, and
\code|p2| cannot have them as well.

The last line of the pattern, line 12, defines that each match should be
represented as a vector containing the set of common movies \code|cms| and the
three persons.  The keyword \code|:distinct| specifies that only distinct
matches should be found.  The reason is that if some clique of three acts in
\(x\)
common movies, there are exactly \(x\)
matches that differ only in the movie \code|m|.  By omitting the movie from the
match representation and specifying that we are only interested in distinct
matches, those duplicates are suppressed.

The last two lines define the action that should be applied on matches.  A new
\code|Clique| element is created that gets assigned the found persons with
their common movies, and the average rating of the common movies.

What has been skipped from explanation until now are the rule's options
specified in line 2.  The usual rule application semantics are that the rule
finds exactly one match (the one it finds first) and then applies its action on
it.  To transform all matches in the model, one would apply the rule
iteratively until no matches can be found anymore.  This behavior is especially
important when a rule application generates new matches or invalidates existing
matches.  However, in this case, neither of both is done.  After a
\code|Clique| has been created, the movie and the three persons are still a
valid match, so calling the rule repeatedly would create multiple \code|Clique|
elements for the same persons.  One could use negative application conditions
in order to forbid matches where there is a \code|Clique| already, but the
\code|:forall| option handles the case in a more concise and efficient way.  It
specifies that when applying the rule \emph{all} matches are searched at once,
and then the action is applied to each of them.  On multi-core systems, FunnyQT
automatically parallelizes this search using Java 7's \emph{ForkJoin} library.

By default, such a \code|:forall| rule returns a sequence of rule application
results (e.g., here a sequence of \code|Clique| elements).  The
\code|:no-result-vec| option specifies that we don't need this sequence which
will make the rule only return the number of matches it has processed in order
to save some memory.


\subsection{Extension Task 1 \& Extension Task 4: Compute Top-15 Couples \&
  Cliques}
\label{sec:ext-task-1:top-15}

The case description demands for the Extension Tasks 1 and 4 the computation of
the top-15 couples and cliques according to the criteria
\begin{compactenum}[(a)]
\item \emph{average rating of common movies}, and
\item \emph{number of common movies}.
\end{compactenum}

Whenever there's a tie between two groups according to the current criterium of
interest, the case description allows for an arbitrary but stable order.  The
FunnyQT solution does a bit more: if there's a tie between two groups for the
current criterium, the respective other criterium is used to cut it.  If that
doesn't suffice, i.e., both groups have the same average rating and number of
common movies, the names of the group's members are compared as a fallback.
Since the person names are unique in the models, there is no chance that no
distinction can be made.

The implementation is quite simple in that the sequence of all \code|Couple|
elements (or all \code|Clique| elements of a given size) are sorted using some
comparator.  Similar to Java, a comparator in Clojure is a function that
receives two objects and returns a negative integer if the first object should
be sorted before the second, a positive integer if the first object should be
sorted after the second item, and zero if both objects are equal with respect
to sorting order.

The comparator for the average rating is shown in the following listing.

\begin{clojurecode}
(defn rating-comparator [a b]
  (compare (emf/eget b :avgRating) (emf/eget a :avgRating)))
\end{clojurecode}

As every comparator, it gets two objects \code|a| and \code|b| (here, two
couples or cliques) and compares them.  \code|compare| is the standard Clojure
comparator which works for objects of any class implementing the
\code|java.lang.Comparable| interface.  Since we compare the average rating of
\code|b| with the average rating of \code|a|, a descending order is defined.

The comparator for the number of common movies is shown below.

\begin{clojurecode*}{firstnumber=3}
(defn common-movies-comparator [a b]
  (compare (.size ^java.util.Collection (emf/eget-raw b :commonMovies))
           (.size ^java.util.Collection (emf/eget-raw a :commonMovies))))
\end{clojurecode*}

Like with the \code|person-count| and \code|movie-count| functions above, we
use type-hints to call the Java method \code|Collection.size()| directly on the
\code|EList| holding the common movies of the two groups \code|a| and \code|b|.

For the next comparator that compares the two groups' actors according to their
names, we need some helper functions first.  Polyfns are FunnyQT's way to
define functions that dispatch polymorphically according to the metamodel type
of its first argument.  First, a polyfn is declared using
\code|declare-polyfn|, and then arbitrary many implementations for different
metamodel types can be added using \code|defpolyfn|.

\begin{clojurecode*}{firstnumber=6}
(poly/declare-polyfn actors [group])

(poly/defpolyfn actors movies.Couple [group]
  [(emf/eget group :p1) (emf/eget group :p2)])

(poly/defpolyfn actors movies.Clique [group]
  (emf/eget-raw group :persons))
\end{clojurecode*}

In line 6, the polyfn \code|actors| with one parameter \code|group| is
declared.  Its intent is to return a collection of all actors and actresses
that are part of the given group.

In lines 7 and 8, one implementation for elements of metamodel type
\code|Couple| is added.  It returns a vector of two elements: the person in the
couple's \code|p1| reference, and the person in the couple's \code|p2|
reference.

Lines 9 and 10 define another implementation for elements of metamodel type
\code|Clique|.  Here, the contents of the group's \code|persons| reference is
returned which already is the collection of all the clique's members.

Using this polyfn, the comparator for ordering groups according to the names of
their members can be defined like shown in the next listing.

\begin{clojurecode*}{firstnumber=13}
(defn names-comparator [a b]
  (compare (str/join ";" (map #(emf/eget % :name) (actors a)))
           (str/join ";" (map #(emf/eget % :name) (actors b)))))
\end{clojurecode*}

It simply compares the strings that result from interleaving each group's actor
names with a semicolon as a separator.  Since here we compare group \code|a|
with group \code|b|, a lexicographically ascending order is achieved.

Until now, there are only three individual comparators, but sorting is always
done with one single comparator.  So the following listing defines a
higher-order comparator, e.g., a function that receives arbitrary many
comparators and returns a new comparator which compares using the given ones.

\begin{clojurecode*}{firstnumber=16}
(defn comparator-combinator [& comparators]
  (fn [a b]
    (loop [cs comparators]
      (if (seq cs)
        (let [r ((first cs) a b)]
          (if (zero? r)
            (recur (rest cs))
            r))
        (u/errorf "%s and %s are incomparable!" a b)))))
\end{clojurecode*}

The function \code|comparator-combinator| returns an anonymous function with
two arguments \code|a| and \code|b|.  This function recurses\footnote{Clojure's
  \code|(loop [<bindings>] ... (recur <newvals>))| is a local tail-recursion.
  \code|loop| establishes bindings just like \code|let|, and \code|recur| jumps
  back to the \code|loop| providing new values for the variables.} over the
given \code|comparators|.  It applies the first one to \code|a| and \code|b|,
and if that results in a non-zero value returns this value.  But if the value
is zero, i.e., no distinction with respect to sorting order can be made with
that comparator, the function recurs and \code|cs| is rebound to the remaining
comparators.  In case all comparators return zero for two given groups, and
error is signalled.

So finally, here are the two top-15 groups functions.

\begin{clojurecode*}{firstnumber=25}
(defn groups-by-avg-rating [groups]
  (sort (comparator-combinator rating-comparator common-movies-comparator names-comparator)
        groups))

(defn groups-by-common-movies [groups]
  (sort (comparator-combinator common-movies-comparator rating-comparator names-comparator)
        groups))
\end{clojurecode*}

\code|groups-by-avg-rating| gets a collection of groups \code|groups| and then
sorts them by the combined comparator first taking the average rating into
account, then the number of common movies, and eventually the names of the
groups' actors if neither of the two former comparators could decide on the two
groups order.

\code|group-by-common-movies| is defined similar except that the
\code|common-movies-comparator| is applied first instead of the
\code|rating-comparator|.

Those where the actually important parts for solving the top-15 tasks.  The
solution contains 39 more lines of code that apply the sorting functions to the
groups in a model, then take the first 15 groups, format the results nicely,
and spit them to files.


\section{Evaluation}
\label{sec:evaluation}

With respect to correctness, the FunnyQT solution computes the exact same
numbers of couples and cliques of various sizes as printed in Table~1 and
Table~2 of the case description.  Also, the top-15 lists are identical for all
models.

Table~\ref{tab:bench-synth} shows the execution times for the synthetic models,
and Table~\ref{tab:bench-imdb} shows the execution times for the IMDb models.
These times include the pattern matching time, the time needed for creating the
\code|Couple| and \code|Clique| elements, and the time needed for setting their
attributes and references including the computation of the average ratings.
The times needed for generating the synthetic models and for loading the IMDb
models are excluded as are the query times for computing the top-15 lists.

The benchmarks were run on a GNU/Linux virtual machine with eight 2.8GHz cores
and 32GB of RAM, 30GB of which were dedicated to the JVM process.

\begin{table}
  \centering
  \begin{tabular}{| r | r | r | r | r |}
    \hline
    \textbf{Model (N)} & \textbf{Couples} & \textbf{3-Cliques} & \textbf{4-Cliques} & \textbf{5-Cliques}\\
    \hline
    1000   &    0.274602528  &  0.349061152  &  0.483500704  &  0.422209712\\
    2000   &    0.547466160  &  0.722435888  &  0.724224848  &  0.628254896\\
    3000   &    0.867141568  &  1.069310832  &  1.028428128  &  0.949939408\\
    4000   &    1.119036992  &  1.437665888  &  1.375804384  &  1.216168544\\
    5000   &    1.406045888  &  1.798415536  &  1.723140656  &  1.502462976\\
    10000  &    2.830416128  &  3.607990864  &  3.482801824  &  3.098102832\\
    50000  &    14.324653712 &  17.972272736 &  17.346692096 &  15.327367392\\
    100000 &    28.300907200 &  36.352164048 &  35.396692384 &  31.440208448\\
    200000 &    57.159804192 &  72.209621472 &  69.233912672 &  60.430165808\\
    \hline
  \end{tabular}
  \caption{Execution times in seconds for the synthetic models}
  \label{tab:bench-synth}
\end{table}

As can be seen in Table~\ref{tab:bench-synth}, the FunnyQT solution scales
completely linearly for the synthetic models which is expected due to their
construction.

\begin{table}
  \centering
  \begin{tabular}{| l | r | r |}
    \hline
    \textbf{Model}                  & \textbf{Couples} & \textbf{3-Cliques}\\
    \hline
    imdb-0005000-49930.movies.bin   &   1.677278992    &        6.957570992\\
    imdb-0010000-98168.movies.bin   &   1.702668160    &        11.333623024\\
    imdb-0030000-207420.movies.bin  &   2.610028032    &        12.507362768\\
    imdb-0045000-299504.movies.bin  &   3.947932624    &        15.714349728\\
    imdb-0065000-404920.movies.bin  &   6.170203664    &        21.243492448\\
    imdb-0085000-499995.movies.bin  &   9.012942560    &        26.388751712\\
    imdb-0130000-709551.movies.bin  &   18.135307008   &        54.709762992\\
    imdb-0200000-1004463.movies.bin &   35.409998560   &        117.833811360\\
    imdb-0340000-1505143.movies.bin &   88.052992592   &        366.832061200\\
    imdb-0495000-2000900.movies.bin &   159.973018224  &        757.280006768\\
    imdb-0660000-2501893.movies.bin &   278.318685728  &        1457.689379247\\
    imdb-all-3257145.movies.bin     &   619.160156640  &        4295.030516512\\
    \hline
  \end{tabular}
  \caption{Execution times in seconds for the IMDb models}
  \label{tab:bench-imdb}
\end{table}

In contrast, the transformation of the real IMDb models requires more effort in
general because they are much more cross-linked.  Whereas in the synthetic
models, every person acts in at most five movies, and every movie has at most
five acting persons, in the complete IMDb model, there are persons acting in up
to 1800 movies, and movies with more than 1200 actors.

The main bottleneck of the FunnyQT transformation is the required memory.  The
generated rules for finding groups of a given size first compute all matches
(each match being represented as a vector containing the persons being members
of the group plus their set of common movies) and then generate one new couple
or clique element for each match.  This means that the original model, all
matches, and also all new elements reside in memory at the same time.

One could sacrifice a bit of performance for better memory-efficiency by not
including the set of common movies already in the matches, but instead
re-compute it in the rule's action creating the couple or clique elements.


\section{Conclusion}
\label{sec:conclusion}

In this paper, the FunnyQT solution to the TTC 2014 Movie Database Case has
been discussed.  It correctly solves all core and extension tasks, and its
performance is quite good due to the fact that FunnyQT is able to perform
pattern matching in parallel on multi-core systems.

Also, the solution is concise.  All in all, it consists of 152 lines of code
(including boilerplace code like namespace definitions) in three source files,
one for the generation of the synthetic test models (30 LOC), one for the
couple and cliques rules (52 LOC), and one for the queries (70 LOC).


\bibliographystyle{eptcs}
\bibliography{ttc-movie-couples}
\end{document}

%%% Local Variables:
%%% mode: latex
%%% TeX-master: t
%%% TeX-engine: pdflatex-shell-escape
%%% End:
